\documentclass[letter,12pt]{article}
\usepackage{amsmath}
\usepackage{graphicx}
\usepackage{glosstex}
\usepackage[utf8]{inputenc}
\usepackage[spanish]{babel}
\usepackage{tikz,stackengine}

%opening
\title{Reporte 02 Evaluación 29 10 2019}
\author{Eduardo Alexis Valencia Dorantes}
\date{30/Octubre/2019}
\begin{document}
	
\maketitle{ASEFE-PBS.py}

\begin{abstract}
Dado que ya tenemos hecho el programa "ASEFE-PBS.py" daremos algunos comentarios sobre el mismo.
\end{abstract}

\section*{La idea}
A partir de un rectángulo de base x y altura 1 definiremos la raíz cuadrada de cualquier número, siempre y cuando este sea mayor a cero.
\section*{El procedimiento}
Empezaremos a definir nuestras igconitas que irán tomando los valores que iremos requiriendo conforme avancemos en el problema.

Ya teniendo nuestras incógnitas podemos proceder especificando cual será nuestra condición inicial. Después de ello tenemos que definir la función que nos dará las condiciones consecutivas a la inicial para calcular la raíz de nuestro primer rectángulo.

Lo único que restaría es pedirle a Python que nos muestre los resultados en forma de tabla; y lo siguiente es hacerlo lo mismo con cualquier rectángulo con base distinta al anterior.
\newpage
\section*{Resultados}
Para $x = 81$
\begin{table}[h!]
	\centering
	\begin{tabular}{ |c|c|c| } 
		\multicolumn{3}{c}{$x = 81$}\\
		\hline
		rectángulo & b & h \\
		\hline
		0& 81	& 1\\
		1& 41.00	& 1.98\\
		2& 21.49	& 3.77\\
		3& 12.63	& 6.41\\
		4&  9.52	& 8.51\\
		5&  9.01	& 8.99\\
		6&  9.00    & 9.00\\
		\hline
	\end{tabular}
	\caption{Rectángulos para $x$ = 81 }
	\label{table:1}
\end{table}

Para $x = 95$
\begin{table}[h!]
	\centering
	\begin{tabular}{ |c|c|c| } 
		\multicolumn{3}{c}{$x = 95$}\\
		\hline
		rectángulo & b & h \\
		\hline
		0& 95	& 1\\
		1&     48.00&  1.98\\
		2&     24.99&  3.80\\
		3&     14.40&  6.60\\
		4&     10.50&  9.05\\
		5&      9.77&  9.72\\
		6&      9.75&  9.75\\
		\hline
	\end{tabular}
	\caption{Rectángulos para $x$ = 95 }
	\label{table:1}
\end{table}
\newpage
Para $x = 0.5$
\begin{table}[h!]
	\centering
	\begin{tabular}{ |c|c|c| } 
		\multicolumn{3}{c}{$x = 0.5$}\\
		\hline
		rectángulo & b & h \\
		\hline
		0& 0.5	& 1\\
		1&      0.75&  0.67\\
		2&      0.71&  0.71\\
		3&      0.71&  0.71\\
		4&      0.71&  0.71\\
		5&      0.71&  0.71\\
		6&      0.71&  0.71\\
		\hline
	\end{tabular}
	\caption{Rectángulos para $x$ = 0.5 }
	\label{table:1}
\end{table}

Para $x = 0.125$
\begin{table}[h!]
	\centering
	\begin{tabular}{ |c|c|c| } 
		\multicolumn{3}{c}{$x = 0.125$}\\
		\hline
		rectángulo & b & h \\
		\hline
		0& 0.125	& 1\\
		1&      0.56&  0.22\\
		2&      0.39&  0.32\\
		3&      0.36&  0.35\\
		4&      0.35&  0.35\\
		5&      0.35&  0.35\\
		6&      0.35&  0.35\\
		\hline
	\end{tabular}
	\caption{Rectángulos para $x$ = 0.125 }
	\label{table:1}
\end{table}

\begin{thebibliography}{0}
	\bibitem[PBS1996]{AlgoritmosPablo}Barrera Sánchez Pablo, Algoritmos Sencillos para Evaluar Funciones Elementales, 1996.
\end{thebibliography}

\end{document}